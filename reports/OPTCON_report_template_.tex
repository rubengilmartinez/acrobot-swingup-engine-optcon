% Template for OPTCON-RL course projects - Acrobot Assignment
\documentclass[a4paper,11pt,oneside]{book}

% Use UTF-8 for easy typing of tildes like 'é' and 'ñ'
\usepackage[utf8]{inputenc}
\usepackage[english]{babel}
\usepackage{amsfonts}
\usepackage{amsmath}
\usepackage{amssymb,color}
\usepackage{cite}
\usepackage{graphicx}
\usepackage{float}
\usepackage[a4paper, total={6in, 8in}]{geometry}

\begin{document}
\pagestyle{myheadings}

%%%%%%%%%%% Cover %%%%%%%%%%%
\thispagestyle{empty}                                        
\begin{center}                                                         
    \vspace{5mm}
    {\LARGE UNIVERSIT\`A DI BOLOGNA} \\                       
      \vspace{5mm}
\end{center}
\begin{center}
  % Ensure this file exists in a 'figs' folder
  \includegraphics[scale=.27]{figs/logo_unibo}
\end{center}
\begin{center}
      \vspace{5mm}
      {\LARGE School of Engineering} \\
        \vspace{3mm}
      {\Large Master Degree in Automation Engineering} \\
      \vspace{20mm}
      {\LARGE Optimal Control and Reinforcement Learning} \\
      \vspace{5mm}{\Large\textbf{OPTIMAL CONTROL OF A GYMNAST ROBOT}}                  
      \vspace{15mm}
\end{center}

\begin{center}                                                                                
     {\large Professor: \textbf{Giuseppe Notarstefano}} \\        
      \vspace{13mm}
\end{center}

\begin{center}
      {\large Students:}\\
\end{center}

\begin{center}
      {\textbf{Rubén Gil Martínez}}\\
\end{center}



\begin{center}
      {\textbf{Guillermo López Pérez}}\\
\end{center}



\begin{center}
\vfill
      {\large Academic year 2025/2026} \\
\end{center}

\newpage
\thispagestyle{empty}

%%%%%%%%%%% Abstract %%%%%%%%%%%%
\chapter*{Abstract}
\thispagestyle{empty}
This project focuses on the optimal control of a planar gymnast robot, modeled as a double pendulum with torque applied only at the hip (Acrobot). We implement trajectory generation using Newton-like algorithms and tracking via LQR and MPC techniques[cite: 37, 51, 61].

\newpage
%%%%%%%%%% Lists %%%%%%%%%%
\tableofcontents 
\thispagestyle{empty}
\newpage
\listoffigures 
\thispagestyle{empty}
\newpage

%%%%%%%%%% Introduction %%%%%%%%%%
\chapter*{Introduction}
\addcontentsline{toc}{chapter}{Introduction}
The goal of this project is to design an optimal trajectory for a planar gymnast robot[cite: 3]. The system consists of two links where actuation is provided only at the second joint (the hip)[cite: 3, 12]. The dynamics involve nonlinear interactions described by mass, Coriolis, and gravity matrices[cite: 13, 16].

\section*{Contributions}
% Mention individual contributions here

%%%%%%%%%% Tasks %%%%%%%%%%

\chapter{Problem Setup and Discretization}
% Task 0: Discretize the dynamics[cite: 23, 24].
In this section, we present the continuous-time dynamics of the Acrobot and the discretization scheme used (e.g., Runge-Kutta 4th order) to transform the system into discrete-time state-space equations[cite: 24, 26].

\chapter{Trajectory Generation}
% Task 1 & 2: Equilibria and Newton-like algorithm[cite: 35, 48].
This chapter covers the computation of system equilibria and the generation of an optimal transition trajectory using a closed-loop Newton-like algorithm[cite: 36, 37]. We also explore trajectory generation for smooth state-input curves[cite: 49].

\chapter{Trajectory Tracking via LQR}
% Task 3: Linearization and LQR[cite: 50].
We linearize the robot dynamics around the reference trajectory and implement an Infinite or Finite Horizon LQR controller to ensure robust tracking under perturbed initial conditions[cite: 51, 54].

\chapter{Trajectory Tracking via MPC}
% Task 4: MPC implementation[cite: 60].
This chapter discusses the implementation of Model Predictive Control (MPC) to track the optimal reference, accounting for system constraints and performance[cite: 61].

\chapter{Results and Animation}
% Task 5: Animation and required plots[cite: 63, 65].
We present the simulation results, including the required semi-logarithmic plots for cost and descent directions[cite: 70, 71]. A link to the Python animation of the gymnast robot is also provided[cite: 64].

%%%%%%%%%% Conclusions %%%%%%%%%%
\chapter*{Conclusions}
\addcontentsline{toc}{chapter}{Conclusions} 
% Summary of findings

%%%%%%%%%% Bibliography %%%%%%%%%%%
% Ensure bibfile.bib exists or use biblatex
\addcontentsline{toc}{chapter}{Bibliography}
\bibliographystyle{plain}
\bibliography{bibfile}

\end{document}