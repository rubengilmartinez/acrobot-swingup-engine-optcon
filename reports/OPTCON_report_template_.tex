% Template for OPTCON-RL course projects - Acrobot Assignment
\documentclass[a4paper,11pt,oneside]{book}

% Use UTF-8 for easy typing of tildes like 'é' and 'ñ'
\usepackage[utf8]{inputenc}
\usepackage[english]{babel}
\usepackage{amsfonts}
\usepackage{amsmath}
\usepackage{amssymb,color}
\usepackage{cite}
\usepackage{graphicx}
\usepackage{float}
\usepackage[a4paper, total={6in, 8in}]{geometry}

\begin{document}
\pagestyle{myheadings}

%%%%%%%%%%% Cover %%%%%%%%%%%
\thispagestyle{empty}                                        
\begin{center}                                                         
    \vspace{5mm}
    {\LARGE UNIVERSIT\`A DI BOLOGNA} \\                       
      \vspace{5mm}
\end{center}
\begin{center}
  % Ensure this file exists in a 'figs' folder
  \includegraphics[scale=.27]{figs/logo_unibo}
\end{center}
\begin{center}
      \vspace{5mm}
      {\LARGE School of Engineering} \\
        \vspace{3mm}
      {\Large Master Degree in Automation Engineering} \\
      \vspace{20mm}
      {\LARGE Optimal Control and Reinforcement Learning} \\
      \vspace{5mm}{\Large\textbf{OPTIMAL CONTROL OF A GYMNAST ROBOT}}                  
      \vspace{15mm}
\end{center}

\begin{center}                                                                                
     {\large Professor: \textbf{Giuseppe Notarstefano}} \\        
      \vspace{13mm}
\end{center}

\begin{center}
      {\large Students:}\\
\end{center}

\begin{center}
      {\textbf{Rubén Gil Martínez}}\\
\end{center}



\begin{center}
      {\textbf{Guillermo López Pérez}}\\
\end{center}



\begin{center}
\vfill
      {\large Academic year 2025/2026} \\
\end{center}

\newpage
\thispagestyle{empty}

%%%%%%%%%%% Abstract %%%%%%%%%%%%
\chapter*{Abstract}
\thispagestyle{empty}
This project focuses on the optimal control of a planar gymnast robot, modeled as a double pendulum with torque applied only at the hip (Acrobot). We implement trajectory generation using Newton-like algorithms and tracking via LQR and MPC techniques[cite: 37, 51, 61].

\newpage
%%%%%%%%%% Lists %%%%%%%%%%
\tableofcontents 
\thispagestyle{empty}
\newpage
\listoffigures 
\thispagestyle{empty}
\newpage

%%%%%%%%%% Introduction %%%%%%%%%%
\chapter*{Introduction}
\addcontentsline{toc}{chapter}{Introduction}
The goal of this project is to design an optimal trajectory for a planar gymnast robot[cite: 3]. The system consists of two links where actuation is provided only at the second joint (the hip)[cite: 3, 12]. The dynamics involve nonlinear interactions described by mass, Coriolis, and gravity matrices[cite: 13, 16].

\section*{Contributions}
% Mention individual contributions here

%%%%%%%%%% Tasks %%%%%%%%%%

\chapter{Problem Setup and Discretization}
\label{chap:setup}

In this chapter, we define the mathematical model of the planar gymnast robot (Acrobot) and the numerical methods employed to simulate its behavior for optimal control purposes.

\section{Continuous-Time Dynamics}
The system is modeled as a planar double pendulum with actuation provided solely at the second joint (the hip). [cite_start]The state of the system is defined as $x = [\theta_1, \theta_2, \dot{\theta}_1, \dot{\theta}_2]^\top$, where $\theta_1$ is the angle of the first link with respect to the vertical, and $\theta_2$ is the relative angle of the second link[cite: 11].

The equations of motion are given in the standard manipulator form:
\begin{equation}
    M(q)\ddot{q} + C(q, \dot{q})\dot{q} + F\dot{q} + G(q) = \begin{bmatrix} 0 \\ \tau \end{bmatrix}
    \label{eq:manipulator}
\end{equation}
[cite_start]where $q = [\theta_1, \theta_2]^\top$ represents the configuration vector and $\tau$ is the control torque applied to the second link[cite: 13].

[cite_start]The system matrices are defined as follows[cite: 16, 17]:
\begin{itemize}
    \item The Mass Matrix $M(q)$, which captures the inertial properties of the links:
    \begin{equation*}
    M(q) = \begin{bmatrix} 
    I_1 + I_2 + m_1 l_{c1}^2 + m_2(l_1^2 + l_{c2}^2 + 2l_1 l_{c2} \cos \theta_2) & I_2 + m_2 l_{c2}(l_1 \cos \theta_2 + l_{c2}) \\
    I_2 + m_2 l_{c2}(l_1 \cos \theta_2 + l_{c2}) & I_2 + m_2 l_{c2}^2
    \end{bmatrix}
    \end{equation*}
    
    \item The Coriolis Matrix $C(q, \dot{q})$, describing forces proportional to velocity squared (centripetal/Coriolis):
    \begin{equation*}
    C(q, \dot{q}) = \begin{bmatrix} 
    -m_2 l_1 l_{c2} \dot{\theta}_2 \sin \theta_2 & -m_2 l_1 l_{c2} (\dot{\theta}_1 + \dot{\theta}_2) \sin \theta_2 \\
    m_2 l_1 l_{c2} \dot{\theta}_1 \sin \theta_2 & 0
    \end{bmatrix}
    \end{equation*}
    
    \item The Gravity Vector $G(q)$, derived from potential energy:
    \begin{equation*}
    G(q) = \begin{bmatrix} 
    g m_1 l_{c1} \sin \theta_1 + g m_2 (l_1 \sin \theta_1 + l_{c2} \sin(\theta_1 + \theta_2)) \\
    g m_2 l_{c2} \sin(\theta_1 + \theta_2)
    \end{bmatrix}
    \end{equation*}
    
    \item The Friction Matrix $F$, modeling viscous damping coefficients $f_1, f_2$:
    \begin{equation*}
    F = \begin{bmatrix} f_1 & 0 \\ 0 & f_2 \end{bmatrix}
    \end{equation*}
\end{itemize}

To simulate the system, we isolate the acceleration $\ddot{q}$ by inverting the mass matrix:
\begin{equation}
    \ddot{q} = M(q)^{-1} \left( \begin{bmatrix} 0 \\ \tau \end{bmatrix} - C(q, \dot{q})\dot{q} - F\dot{q} - G(q) \right)
\end{equation}
[cite_start]For this project, we utilize \textbf{Parameter Set 3}, characterized by heavier links ($m_{1,2}=1.5$ kg) and higher inertia ($I_{1,2}=2$ kg$\cdot$m$^2$) compared to the baseline[cite: 21].

\section{Discretization Scheme}
To employ discrete-time optimal control algorithms, we discretize the continuous dynamics $\dot{x} = f_c(x, u)$ into a mapping $x_{k+1} = f_d(x_k, u_k)$. [cite_start]We employ the \textbf{Runge-Kutta 4th Order (RK4)} method to ensure numerical stability and high accuracy, which is critical for the convergence of Newton-type algorithms[cite: 26].

[cite_start]Given a time step $d_t$, the update rule is defined as[cite: 29, 30, 31, 32, 33]:
\begin{align}
    k_1 &= f_c(x_k, u_k) \nonumber \\
    k_2 &= f_c(x_k + \frac{d_t}{2} k_1, u_k) \nonumber \\
    k_3 &= f_c(x_k + \frac{d_t}{2} k_2, u_k) \nonumber \\
    k_4 &= f_c(x_k + d_t k_3, u_k) \nonumber \\
    x_{k+1} &= x_k + \frac{d_t}{6} (k_1 + 2k_2 + 2k_3 + k_4)
\end{align}
This discretized model serves as the foundation for the trajectory generation and tracking tasks in subsequent chapters.







\chapter{Trajectory Generation}
\label{chap:generation}

This chapter addresses the generation of optimal trajectories to maneuver the Acrobot from a stable hanging position to an unstable inverted position. [cite_start]We employ a Newton-type method for optimal control (Differential Dynamic Programming / Iterative LQR) to solve the nonlinear optimization problem[cite: 37].

\section{Computation of Equilibria}
The first step involves identifying the start and end configurations. [cite_start]An equilibrium point $(x_{eq}, u_{eq})$ satisfies the condition $f_c(x_{eq}, u_{eq}) = 0$ (zero acceleration and zero velocity)[cite: 36].
Using a numerical root-finding routine, we computed the following equilibria:
\begin{itemize}
    \item \textbf{Stable Equilibrium (Down):} $x_{eq1} = [0, 0, 0, 0]^\top$. The robot hangs vertically under gravity.
    \item \textbf{Unstable Equilibrium (Up):} $x_{eq2} = [\pi, 0, 0, 0]^\top$. The robot is balanced perfectly upright.
\end{itemize}
Note that while the holding torque for the perfectly upright position is theoretically zero due to alignment with the gravity vector, the configuration is naturally unstable.

\section{Task 1: Step Reference Trajectory}
[cite_start]For the initial trajectory generation task, we define a reference curve composed of two long constant segments with a sharp transition, as suggested in the problem statement[cite: 39].
\begin{equation}
    x_{ref}(t) = \begin{cases} 
    x_{eq1} & 0 \le t < T/2 \\
    x_{eq2} & T/2 \le t \le T 
    \end{cases}
\end{equation}
See Figure \ref{fig:ref_step} for the visualization of this reference.

\subsection{Newton-type Algorithm Implementation}
We implemented the \textbf{Closed-Loop Newton's Method} (Iterative LQR). This algorithm iteratively improves the control sequence $u$ by minimizing a quadratic approximation of the cost function around the current trajectory.
Crucially, the algorithm utilizes a ``closed-loop'' rollout policy during the line search phase:
\begin{equation}
    u_{new}(t) = u_{old}(t) + \alpha k(t) + K(t)(x_{new}(t) - x_{old}(t))
\end{equation}
where $K(t)$ is the feedback gain computed during the backward pass. [cite_start]This feedback term stabilizes the rollout of the unstable Acrobot dynamics, allowing the optimizer to converge significantly faster than standard open-loop shooting methods[cite: 37].

To ensure robust descent, we implemented an \textbf{Armijo Line Search} mechanism. This verifies that the updated trajectory provides a sufficient decrease in the cost function $J$ relative to the step size $\alpha$ and the directional derivative.

% --- PLOT PLACEHOLDERS TASK 1 ---
\begin{figure}[H]
    \centering
    % \includegraphics[width=0.8\textwidth]{figs/task1_optimal_traj}
    \caption{Task 1: Optimal Trajectory vs. Step Reference. The robot performs a swing-up maneuver to match the step change.}
    \label{fig:task1_traj}
\end{figure}

\begin{figure}[H]
    \centering
    % \includegraphics[width=0.8\textwidth]{figs/task1_intermediate}
    \caption{Task 1: Intermediate trajectories showing the iterative convergence of the Newton solver.}
    \label{fig:task1_inter}
\end{figure}

\begin{figure}[H]
    \centering
    % \includegraphics[width=0.8\textwidth]{figs/task1_armijo}
    \caption{Armijo descent direction analysis and cost evolution (semi-log scale).}
    \label{fig:task1_cost}
\end{figure}

\section{Task 2: Natural Reference Trajectory}
[cite_start]In Task 2, we are required to perform the trajectory generation on a ``desired (smooth) state-input curve''[cite: 49].
We observed that the sharp step reference in Task 1 creates a conflict between the cost function (which penalizes deviation from zero) and the system physics (which requires swinging to gain momentum).

To resolve this, we designed a \textbf{Physically Inspired Reference} using a simplified model approximation:
\begin{enumerate}
    \item \textbf{Phase 1 (Energy Pumping):} We modeled the system as a simple pendulum to derive its natural frequency $\omega_n \approx \sqrt{g/L_{eff}}$. We injected sinusoidal oscillations at this frequency into the reference for the first half of the horizon. This encourages the solver to ``pump'' energy rather than fight the cost function.
    \item \textbf{Phase 2 (Swing Up):} We utilized a smooth sigmoid function to transition the reference from the oscillating bottom state to the inverted equilibrium $x_{eq2}$, avoiding the infinite derivatives associated with a step change.
\end{enumerate}

This approach yields a reference trajectory that is more consistent with the underactuated dynamics of the system, facilitating easier tracking and convergence.

% --- PLOT PLACEHOLDERS TASK 2 ---
\begin{figure}[H]
    \centering
    % \includegraphics[width=0.8\textwidth]{figs/task2_optimal_traj}
    \caption{Task 2: Optimal Trajectory tracking the ``Natural'' smooth reference (Oscillations + Sigmoid).}
    \label{fig:task2_traj}
\end{figure}

\begin{figure}[H]
    \centering
    % \includegraphics[width=0.8\textwidth]{figs/task2_cost_log}
    \caption{Task 2: Cost evolution along iterations (semi-logarithmic scale).}
    \label{fig:task2_cost}
\end{figure}




\chapter{Trajectory Tracking via LQR}
% Task 3: Linearization and LQR[cite: 50].
We linearize the robot dynamics around the reference trajectory and implement an Infinite or Finite Horizon LQR controller to ensure robust tracking under perturbed initial conditions[cite: 51, 54].

\chapter{Trajectory Tracking via MPC}
% Task 4: MPC implementation[cite: 60].
This chapter discusses the implementation of Model Predictive Control (MPC) to track the optimal reference, accounting for system constraints and performance[cite: 61].

\chapter{Results and Animation}
% Task 5: Animation and required plots[cite: 63, 65].
We present the simulation results, including the required semi-logarithmic plots for cost and descent directions[cite: 70, 71]. A link to the Python animation of the gymnast robot is also provided[cite: 64].

%%%%%%%%%% Conclusions %%%%%%%%%%
\chapter*{Conclusions}
\addcontentsline{toc}{chapter}{Conclusions} 
% Summary of findings

%%%%%%%%%% Bibliography %%%%%%%%%%%
% Ensure bibfile.bib exists or use biblatex
\addcontentsline{toc}{chapter}{Bibliography}
\bibliographystyle{plain}
\bibliography{bibfile}

\end{document}